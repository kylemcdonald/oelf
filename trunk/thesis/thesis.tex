\documentclass{thesis}
\usepackage{graphicx}
\usepackage{hyperref}
\usepackage{url}

\thesistitle{\bf Only Everything Lasts Forever}        
\author{Kyle McDonald}        
\degree{Master of Fine Arts}
\department{Electronic Arts}
\projadviser{Curtis Bahn}
\coprojadviser{Michael Century}
\cocoprojadviser{Shawn Lawson}
%\cothadviser{First co-adviser} %if needed
%\cocothadviser{Second co-adviser} % if needed
%  For a masters project use \projadviser instead of \thadviser, 
%  and \coprojadviser and \cocoprojadviser if needed. 
\submitdate{April 2010\\(For Graduation May 2010)}        
\copyrightyear{2010}
      
\begin{document} 
\titlepage
\tableofcontents
%\listoftables % required if there are tables
%\listoffigures % required if there are figures

\specialhead{Acknowledgments}
I would like to thank...

\specialhead{Abstract}
Only Everything Lasts Forever is a sound composition containing every sound we can distinguish. It explores the social and political associations of sound representation along with the psychology and philosophy of noise and emptiness.

\chapter{Introduction}

Noise is a human invention. Not only an invention, but a necessity. Noise is misunderstanding and acontextuality. Without humanity, there is no misunderstanding to be had, and thus no noise.

Music is unique in that it represents a subset of non-noise that is historically and biologically informed.

Only Everything Lasts Forever is a sound composition containing every sound we can distinguish. It explores the social and political associations of sound representation along with the psychology and philosophy of noise and emptiness.

\chapter{The Spirit of Noise}
\section{Epistemology: Correspondence vs. Coherence}
	\cite{Blackburn07}\cite{seti_about_????}\cite{david_correspondence_????}\cite{david_horvitz_flickr:_????}
	\cite{david_horvitz_flickr:_????-1}\cite{young_coherence_????}
\section{Philosophy: Interdependent Arising}
	\cite{Hofstadter07}\cite{Koller01}\cite{erik_thiele_tempest_????}\cite{francesco_vianello_reality_????}
	\cite{w._wayt_gibbs_hackers_2009}
\section{Psychology and Nature: Gestalt, Illusion, and Patterns}
	\cite{markowsky_misconceptions_1992}
	\cite{Moore07}\cite{Doczi81}\cite{Hofstadter01}\cite{robin_mckie_secret_2004}
	\cite{alan_dunning_paul_woodrow_and_morley_hollenberg_einsteins_2008}\cite{alexander_bogomolny_kanizsa_????}
	\cite{brian_dunning_facemars_2008}\cite{dan_paluska_holy_2005}\cite{de_lapparent_slice_1986}
	\cite{jochem_van_der_spek_no_2001}\cite{jonathan_feinberg_haiku_????}\cite{michael_bach_dalmatian_2002}
	\cite{michael_m._ross_natural_2007}\cite{padovan_proportion_1999}\cite{rips_equidistant_1994}
	\cite{weisstein_prime_????}\cite{boston.com_religious_????}
\section{Music: Acontextual Sound}
	\cite{Bruen04}\cite{Vannort06}\cite{Attali85}\cite{Cage61}\cite{Sangild04}\cite{Cascone00}
	\cite{Hegarty02}\cite{Kahn01}\cite{Russolo04}
	
\chapter{Everything Art}
\section{The Historical ``End of Art''}
	\cite{Wright09}
	Wright describes systems artists as the ``the last programmers before the digital computer made that practice synonymous with its own functioning.'' At the boundary of systems art and modern-day programming were pieces like Paul Brown's ``Lifemods'' in the late 70s:
	
	\begin{quote}
	Visually the work was formally more sophisticated and texturally richer than previous Systems work, but most striking was the sheer quantity of graphics and audio that was now being produced, something approaching a continuous torrent of sensory data.
	\end{quote}
	
	\cite{moma_kazimir_2006}
	``White on White'' by Kazimir Malevich is totally non-representational: a skewed white square against a white background. It was painted in 1918, five years after his form-oriented ``Black Square.'' Leah Dickerman at MoMA says of ``White on White'':
	
	\begin{quote}
	White is a way of taking away--minimizing--color itself, and actually focusing particularly on the material of painting.
	\end{quote}
	
	``White on White'' is not about painting in general, but specifically about Malevich's painting: you see his brush strokes. He described his art as ``Suprematist,'' focusing on ``the supremacy of pure feeling or perception in the pictorial arts.'' From an evolutionary perspective, Suprematism would imply representationalism in that our perceptions are attuned to nature. Malevich takes a more spiritual and idealistic route, dismissing the feelings associated with ``real objects''. He explains in ``From Cubism and Futurism to Suprematism'' that representationalism (``The transferring of real objects onto canvas'') is copying, and a genuine act of creation only occurs when an artist's ``pictures have nothing in common with nature.'' He maintains that the medium itself is the most important part of painting:
	
	\begin{quote}
	For art is the ability to construct, not on the interrelation of form and color, and not on an aesthetic basis of beauty in composition, but on the basis of weight speed and the direction of movement. [...] Color and texture in painting are ends in themselves. They are the essence of painting, but this essence has always been destroyed by the subject.
	\end{quote}
	
	It's interesting that these first Suprematist works of Malevich (``Black Square'', ``Red Square'', ``White on White'') used squares, which he called a ``zero form.'' More geometrically elegant is the circle. More intuitive to paint is the line, or even the point. But he focused on the square (and later, rectangles), in a strange foreshadowing of raster images.
	
	\cite{moma_rodchenko_1998}
	In 1921 Alexander Rodchenko first exhibited his three canvases, ``Pure Red Color, Pure Yellow Color, and Pure Blue Color.'' These three canvases were the first true monochromes. He later said of them:
	
	\begin{quote}
	I reduced painting to its logical conclusion and exhibited three canvases: red, blue and yellow. I affirmed: it's all over. Basic colors. Every plane is a plane and there is to be no representation.
	\end{quote}
	
	Rodchenko deals with painting here directly as a medium: the core of painting with respect to color is the principle of primaries in pigment-based subtractive color theory.
	
\section{Permutation and Enumeration}
	pppd\cite{kyle_mcdonald_pppd_2009} is an indirect approach to enumeration via a random walk. It draws on computability theory\cite{boolos_computability_2002} and the idea of a Turing-complete system. Turing-completeness refers to the ability of a system to compute anything that can be computed. pppd uses a very simple esoteric programming language known as P'' (``P prime prime'') to accomplish this. While practical programming languages are full of statements like \verb!z = x + y!, \verb!println(z)!, and \verb!for(int i = 0; i < n; i++)!, P'' breaks computing down to its essentials: moving a pointer through memory, changing memory, and looping--requiring only six commands, sometimes written as [, ], +, -, < and >. Because P'' is Turing-complete, any program you could write with a language like C++ or Java could also be accomplished with P'' (albeit in a significantly slower and heavily obfuscated manner). Because any possible program can be represented, this means that any possible digital image or sound can also be written to the program's memory space. pppd generates random code, fitting the minor syntactical requirement of balanced brackets, and then visualizes and sonifies the memory space while running the code. In theory, not only can every image and sound be composed, but every compressed representation as well--along with the decompression algorithm. In practice, pppd produces very regular sequences of flashing colors and glitchy sounds. This acts a revelation of the nature of P'': it's likes to representing high-contrast memory sequences within constrained memory regions. Using another language would generate very different results.
	
	My nandhopper project\cite{kyle_mcdonald_nandhopper_2008} also draws on enumeration via directed random walks. The NAND is a binary logical operation, meaning it takes two inputs that are either true or false and returns a result of true or false. The NAND has the unusual property of being able to represent any possible logical operation on any number of inputs. In other words, while you might normally express a logical statement using multiple operators like IF, AND, OR, and NOT, they can all be reduced to statements that only use NAND. Practically, this means that any possible digital sound synthesis circuit can be represented using only NAND gates.
	
	I first came across NAND gates when exploring capacitive sensing. NAND gates are useful for constructing feedback loops that vary their frequency with respect to capacitance to ground.\footnote{Using a Schmitt trigger NAND like the 4093 you can use one input as an on-off switch while tying the output to the other input via a resistor. The feedback input is also tied to ground via a capacitor. This is a very basic RC circuit.} When I realized that NANDs could be used to simulate every other chip, I started exploring various NAND configurations and ways of connecting multiple NAND circuits. I developed three different configurations for an experimental filmmaker\footnote{Mexican filmmaker Diego Delmar, alias ``Macushi'' \url{http://www.knock.com.mx/}} based on intuitive explorations of these configurations. I published these designs along with detailed instructions on the tutorial-sharing site Instructables.com. In Spring of 2009 I developed a close connection to a complex variation on the NAND-synth that was developed for a performance context, integrating capacitive, resistive, and illumination sensing in a massively entangled feedback circuit. I see these as the first steps in a more general exploration of NAND-synthesis. I imagine a massive matrix of reconfigurable NAND circuits that can be rearranged during performance: constantly responding the the changing flow of logic, able to simulate any imaginable digital synthesizer in a very practical sense, complex enough to be interesting, but deterministic enough to be learned.
	
	Borges' short stores ``The Aleph''\cite{borges_aleph_2004} and ``The Library of Babel''\cite{borges_library_2000} are two examples from literature dealing very directly with the themes of ``everything'' and enumeration. The titular subject of ``The Aleph'' is a visual phenomena: a point in space from which you may see all other points in space, ``the only place on earth where all places are -- seen from every angle, each standing clear, without any confusion or blending.'' There is a very interesting note near the end about a sonic corollary:
	
	\begin{quote}
	The Faithful who gather at the mosque of Amr, in Cairo, are acquainted with the fact that the entire universe lies inside one of the stone pillars that ring its central court... No one, of course, can actually see it, but those who lay an ear against the surface tell that after some short while they perceive its busy hum... The mosque dates from the seventh century; the pillars come from other temples of pre-Islamic religions, since, as ibn-Khaldun has written: ``In nations founded by nomads, the aid of foreigners is essential in all concerning masonry.''
	\end{quote}
	
	Carlos Argentino Daneri is a writer who discovers the Aleph in his basement, and is using it to produce an epic poem titled ``The Earth,'' describing everything in detail. While explaining the structure and content of the poem, he links it to scripture with all of its ``enumeration, congeries, and conglomeration.'' In a postscript to the story, the fictionalized Borges mentions the origin of the Aleph's name:
	
	\begin{quote}
	As is well known, the Aleph is the first letter of the Hebrew alphabet. Its use for the strange sphere in my story may not be accidental. For the Kabbalah, the letter stands for the En Soph, the pure and boundless godhead; it is also said that it takes the shape of a man pointing to both heaven and earth, in order to show that the lower world is the map and mirror of the higher...
	\end{quote}
	
	In Kabbalah, there is also an intricate internally consistent numerological interpretation to the aleph character.\footnote{\url{http://gnosticteachings.org/courses/alphabet-of-kabbalah/01-aleph}} The aleph is seen as representing the Kabbalistic Holy Trinity, and this ``explains'' the facts that: the character itself represents the number 1, or unity, and appears at the beginning of the Hebrew aleph-bet; the spelling of its name consists of three characters--the Trinity; finally, the values of these characters--80, 30, 1--sum to 111, or three alephs. The aleph is also associated with the breath of life--an interesting connection given the ``Aum'' of Hinduism that may also represent breath, unity, and the creative force of the universe.
	
	In ``The Library of Babel'', the main character tells of the unusual universe in which he lives: floor after floor of interconnected hexagonal rooms, lined with rows of books on shelves. These shelves are said to contain all the possible 410-page books (40 lines per page, 80 characters per line, from an alphabet of 25 characters). He outlines what this implies regarding the contents of the books:
	
	\begin{quote}
	Everything: the minutely detailed history of the future, the archangels' autobiographies, the faithful catalogs of the Library, thousands and thousands of false catalogs, the demonstration of the fallacy of those catalogs, the demonstration of the fallacy of the true catalog, the Gnostic gospel of Basilides, the commentary on that gospel, the commentary on the commentary on that gospel, the true story of your death, the translation of every book in all languages, the interpolations of every book in all books. 
	\end{quote}
	
	Responding to critics who claim that the library can only ``affirm, negate and confuse everything like a delirious divinity'', he says:
	
	\begin{quote}
	In truth, the Library includes all verbal structures, all variations permitted by the twenty-five orthographical symbols, but not a single example of absolute nonsense. [...] I cannot combine some characters, ``dhcmrlchtdj,'' which the divine Library has not foreseen and which in one of its secret tongues do not contain a terrible meaning. [...] The Library is unlimited and cyclical. If an eternal traveler were to cross it in any direction, after centuries he would see that the same volumes were repeated in the same disorder (which, thus repeated, would be an order: the Order).
	\end{quote}
	
	No book can be nonsense precisely because it exists in the context of an alphabet, in the context of a book, and within the Order of the Library itself. This context is the meaningfulness. Furthermore, in this self-contextualization through enumeration emerges the realization that every possible language is represented, and that this offers another degree of context.
	
	In generative visual art there exists an obvious manifestation to Borges' Library. Our 25-letter alphabet is substituted with a 2-letter alphabet of binary, the lines of text becomes rows, and the page becomes an image. Represented so precisely, the computer can steadily enumerate every possible combination of pixels in this image.
	
	The best known manifestation of this idea is ``Every Icon'' by John F. Simon Jr.\cite{john_f._simon_jr._every_????}\cite{john_f._simon_jr._given:32_1997}\cite{matthew_mirapaul_in_1997}	
	
	\cite{jim_campbell_end_1996}
	\cite{leander_seige_imagen_????}
	\cite{leonardo_solaas_magic_????}
	\cite{sintron_gods_2003}
	
	\cite{michael_aschauer_8-bit_????}
	\cite{nattiez_music_1990}
	\cite{remko_scha_every_2001}
	\cite{tomczak_all_2009}
	\cite{tomczak_hardware-based_2009}
	\cite{alexander_christiaan_jacob_allrgb_2008}
	\cite{allan_mccollum_shapes_2006}
	\cite{jem_finer_longplayer_????}
	\cite{john_cage_as_????}
	\cite{paul_slocum_pi_2007}
	\cite{brian_whitman_eigenradio_2005}
	\cite{keith_f._lynch_converting_????}
	\cite{tom_johnson_liner_1999}
	\cite{christian_scheib_statics_????}
	
	What is the connection between enumeration and indeterminacy? They are both about sampling a large space, but one is approached in a compositional manner while the other is meta-compositional and imitative of nature.
	
	ASLAP could be connected to Cage's comment, ``I'm trying to find a way to make music that does not depend on time.''
	
\section{Empty Art}

	\cite{larry_j_solomon_sounds_1998}
	John Cage's 4'33'' is the perfect example of ``empty art'' in the context of sonic art. Both 4'33'' and OELF have an emphasis on allowing the listener to trust themself. But 4'33'' is primarily concerned with the human experience while OELF is situated in the context of an extra-human ``nature''.
	
	Why were people so mad at Cage for 4'33''? Being presented in the context of a musical performance, he was associating his musical work of silence with the musical work of others. Empty work--4'33'', Duchamp's ready-mades--destroy the artist by elevating the individual. Historically the Western artist has acted as a sort of aesthetic-ubermensch, creating a framework of perception for their audience: a framework rooted in religious interpretation and self-expression, before and after the Renaissance respectively. Without bad faith comes existential angst. Cage seemed to be somewhat unaware of this effect, mentioning:
	
\begin{quote}
	Many people in our society now go around the streets and in the buses and so forth playing radios with earphones on and they don't hear the world around them. They hear only what they have chosen to hear. I can't understand why they cut themselves off from that rich experience which is free.
\end{quote}

	Not everyone is comfortable determining their existence and asserting their freedom.
	
	In 1957, with the avant-garde in full force exploring chance operations, graphic scores, improvisation, and atonal composition, Cage writes, ``Try as we may to make a silence, we cannot, one need not fear for the future of music.''  techniques describing ``the future of music.'' The irony of this statement is that no one was worried about just making more sounds. Cage is responding to the fear of these developments with an encouragement to open our ears: as long as we can't make silence, there is music to be made.
	
	Cage defines silence as ``giving up of intention.'' As we are the only beings capable of giving up intention, it follows that we create not music, but silence. Cage goes on, saying ``Music is continuous. It is only we who turn away.'' Some expositions of related terms:
		
\begin{enumerate}
	\item Cage sees music and silence as exclusive, distinguished only by our intention/attention.
	\item I'm considering noise and non-noise as exclusive, with noise being non-contextual sounds and non-noise being contextual sounds. Music represents a subset of contextual sounds. Noise doesn't exist without us, but neither does music. The problem with Cage's separation is sounds that you are fully attentive to but refuse to understand as musical.
\end{enumerate}

	Alphonse Allais' white painting ``Anaemic Young Girls Going to Their First Communion through a Blizzard'', black painting ``Negroes Fighting in a Cave at Night'' and 24 empty measures in ``Funeral March''. People weren't ready yet, these pieces are rarely mentioned though they are just as conceptually rich as Rauschenberg's white paintings, the Russian painters that preceded him, and Cage's 4'33''.
	
	Why the strong association with death--``Funeral March'', ``The Death of Painting'', ``The End''? Did the artists themselves have the same response to their work as Cage's audience to 4'33''? That without an absolutist correspondence-based account for art, there could be no more art? Yes, there seems to be significant difference between these approaches to emptiness. ``Funeral March'' wasn't just silent in Cage's sense, but it was composed for a deaf man. 4'33'' cannot be audited by a deaf person, as it is about embracing sound. Cage even described 4'33'' as ``the beginning of music,'' clearly seeing the matter in the opposite light.
	
	Rauschenberg's white paintings gave Cage ``permission'' to write 4'33''. The white paintings offer a significant metaphor for understanding 4'33'': not as ``empty'', but as ``mirrors.'' Perhaps ``Mirror Art'' is more fitting than ``Empty Art'' when describing this collection of pieces.
	
	Cage tied this realization of music's nature, intentional sound, to music's function: a process of discovery, becoming more attentive to sounds. When describing noise and non-noise, I understand neither in a functional sense.

\chapter{Glitch Art}

\section{Lossless and Lossy Compression}
	Lossless compression refers to a method of rewriting information. It takes a long signal of data with low information content and rewrites it as a short signal with high information content. In information theory, when a signal looks very noisy and random there is high information content, and when there is obvious structure there is low information content. At first, it might seem like the idea of ``information'' in information theory is almost completely opposite our normal intuition for ``information''.
	
	To relate our intuition to the formal definition, let's consider the question: how much is communicated by each symbol in a stream? Here is a binary description of a coin toss event with one unknown symbol indicated by an asterisk: 100*011. Now consider the same event described in English with many unknown symbols: h**ds t**** t**ls **a*s t*i*s he**s **a**. The fact that we can remove so many symbols from the English description while maintaining the integrity of the message means that each symbol individually communicates a small amount of information. On the other hand, removing a single symbol from the compressed binary representation destroys an entire toss.
	
	Lossy compression also has the property of reducing the size of the signal, but it makes a concession: a signal that is lossily compressed can not be used to perfectly recover the original signal. Lossy compression relies on the fact that some data is more relevant to the signal's general message than other data. For example, given a recording of human speech, we can discard anything above 8 kHz and still get the general message. Depending on the representation, that low pass filter could allow us to halve the size of the compressed signal.
	
	While lossless compression is essentially applied math--dependent primarily on the assumption that various media can be represented digitally and therefore analyzed for redundant information like any other binary data--lossy compression represents a significantly different interpretation of the digital signal. Lossless compression algorithms are purely syntactic, but lossy compression considers the signal's semantics in the context of the receiver. Lossless compression assumes only that a medium can be digitized, but lossy compression goes one step further and defines an associated space for the medium. In theory this is space is directly related to human perception, but in practice it is standardized by small groups of specialists.
	
	This approach to lossily transcribing sound is much more akin to the shorthand used by ethnographers and composers than it is to vinyl, magnetic tape, or PCM format audio in that it treats the audio as a perceptual stimuli rather than a physical event.
	
	A great example of exploiting this phenomena in the visual field is the Monochrome works by RYbN\footnote{\url{http://www.cimatics.com/festival2008/festival/performances/aka_pfmonochrome.html}} where ``black'' video is played in complete darkness, revealing biases of the encoding/decoding process; or the 20 kbps music label\footnote{\url{http://20kbps.sofapause.ch/}}, focused on releasing the lowest fidelity MP3 encoded music.
	
\section{Databending}
	\cite{indianropeburn_databenders_????}
	The databenders group on Yahoo was started on May 5th, 2001, now described as ``Sound Synthesis using Raw Data. Also including discussions of cd-bending, data-to-image, image-to-sound and other related techniques.'' It was founded by Robert Green, who now runs Alien-Devices, which ``has been making professional quality modified and circuit bent instruments for over a decade'' http://alien-devices.com/ The group saw a lot of interest in early 2002, steadily declining through 2006. More than a thousand messages document various experiments in renaming files and using ``raw'' formats for converting between sounds and images. It's mostly practical discussion about which software to use, settings that sound or look interesting, and techniques for post-processing found data. Writing custom software happened, but was less common than using off the shelf software in unconventional and unintended ways.
	
	The databenders stumbled across a number of basic practical observations regarding how different formats represent visual and sonic information. This understanding of lower level structures guided their explorations, sometimes crossing over into glitching rather than transcoding, but always revealing something about the underlying medium itself.
		
	\cite{cory_arcangel_photoshop_2009}
	Cory Arcangel's gradient works explore the influence our tools have on us. The use of the gradient and smudge tools have significantly affected design work of the last few decades. Using only those tools, he constructs an image that is too familiar and repulsively simple.
	
	\cite{media_art_net_media_2010}
	Yasunao Tone's ``Wounded CDs'' work represents a glitch art that explores its own medium. ``Wounded Man'yo'' specifically represents the culmination of a number of transcoding and glitching practices. Tone starts with the ancient Chinese Man'yo texts, and selects pictures that correspond to each Chinese symbol. These pictures are then irreversibly transcoded: each row and column has the histogram calculated, and these are arranged in a linear order. This is done for every image, and resulting signal is burned onto a CD. The CD is then wounded with scratches and other modifications. The resulting sound might be considered an exposition of two representational modalities: the histogram-based image representation, followed by the CD-based sound representation. Because histogram-based image encoding is non-standard, we hear the result primarily as noise, with the exceptions of the CD glitches that reveal the structure CD/CD player system.
	
	Glitch art as a method for exploring a medium is something like musical improvisation in an unusual space. Before you begin, there is the ``silence'' of the space: the sound of 4'33'' or light of a white canvas. With your first sonic gesture you create a wave that is interpreted and reflected back to you by the room, corrupting the silence and revealing something about the rooms structure and biases. In the same way, glitch art is sometimes about making a gesture within a well defined medium: corrupting the natural content of a file in order to reveal its structure.
	
	Cage mentions:

\begin{quote}
...since I can't hear [music] while I'm writing it, I'm able to write something that I've never heard before...
\end{quote}

	This is Cage acknowledging he was one of many composers more interested in hearing notation than composing specific sounds.

	\cite{liminalmike_flickr:glitch_????}
	The Glitch Art Pool on the image-sharing website Flickr has been around since (insert date). It represents a significant body of collective recognition and awareness of glitches. New contributions are regularly submitted by individuals who, in the moment of a system's failure, recognize the glitch as beautiful and desire to share it. The Glitch Art Pool is relevant in that it represents the practical awareness of noise and the structure of encoding/decoding systems.
	
\section{Datamoshing}
	\cite{!mediengruppe_bitnik_and_sven_knig_download_????}
	``Download Finished'' automatically downloads movies from peer to peer filesharing networks, ``transmogrifies'' them, and redistributes the modified versions. They describe it as ``found art'' and claim that their ``transmogrified'' films ``belong to everyone.'' Their transformation is very important to the result, as they feel it will ``reveal the nature of the found footage files as a collaborative work with a very complex data structure.'' Their mode of glitching (removing key frames) is less perceptually oriented than systems oriented:
	
	\begin{quote}
	A film found in a filesharing network is the sum of $1$ the original film, $2$ the work of the mathematicians who laid the theoretical foundations for $3$ the programmers who designed the encoding software / the codec and $4$ the file sharer who finally uses all that software to intentionally make the $5$ film widely available. The processes behind $2-4$ usually stay invisible, leading to the wrong assumption that $1=5$. DOWNLOAD FINISHED transformes[sic] $5$ such, that the processes behind steps $2-4$ become visibile[sic] and show that films found in file sharing networks are actually collaborative works.
	\end{quote}
	
	\cite{evan_meaney_ceibas:_2008}
	``The Ceibas Cycle'' by Evan Meaney approaches digital video as a container for a memory that will steadily fade over time. The work is relevant in that he addresses 27 different codec/container combinations and glitches them all in a similar way: substituting the ASCII representation of their data with some plain text. The resulting videos act as a record of the format's biases and assumptions about perception: some display optical flow artifacts, others quickly strobe magenta when color information goes missing.
	
	My own work, ``Future Fragments,'' was an exploration of glitch-collaboration similar to ``The Ceibas Cycle.'' I described it as:
	
	\begin{quote}
	An anti-time-capsule: quotes from seven fellow art students, transcribed phonetically and encoded as colors. Prints of these colors were carried by the artists for a summer. Five returned. Two extra prints were accidentally intercepted and also returned. Decoded back into phonemes and re-formed into words, each text offers an indirect account of their respective journeys.\footnote{\url{http://www.flickr.com/photos/kylemcdonald/sets/72157608915887288/}}
	\end{quote}
	
	A significant portion of this piece was the time spent developing a meaningful abstract encoding for phonemes. Just as a lossily encoded movie will yield something perceptually similar to the original even in the face of glitch, I developed a lossless encoding that was perceptually informed--in order to degrade in a perceptually relevant way. Bearing a vague resemblance to our mythologies and parables around the world that have gone through steady glitch (telephone/Chinese whispers and ''I am Sitting in a Room'') and co-evolved to reflect the structure of our collective mind rather than any specific historical event.
	
	\cite{imbecil_mpeg_2004}
	MPF (``MPeg Fucker'') by imbecil follows the general tradition of glitch-alike by making ASCII substitutions in the file. For example, substituting `a' (0110 0001) with `9' (0011 1001). Two scripts are provided for glitching files in slightly different ways, distinguished only by the choices of letter combinations for substitution. There is a bit of structural information represented here in that the substitutions themselves are fixed. The substitutions aren't made at random, but some favorites are encoded and shared.
	
	\cite{john_michael_boling_rhizome_????}
	John Michael Boling provided an open forum for discussion of the ``pixel bleed'' effect on the Rhizome blog, and saw a number of responses noting various artists who have worked with the technique. It starts with the advent of lossy compression itself, developing into the most recent mainstream Kanye West and Chairlift music videos. This discussion moves away from the politics and biases of the compression, and towards the ownership and connotations of the technique.
	
	\cite{nikolai_trunichkin_and_dr._dmitriy_vatolin_crazy_????}
	The Moscow State University Graphics and Media Lab maintains this interesting page of glitches they've stumbled across while developing their filter library (a collection of small visual processing modules used for research purposes). The unusual thing here is that they've made remarks on exactly which part of the algorithm has gone astray during the implementation. This is the other end of the Databender's Wordpad-mod scene: instead of making uninformed and intuitive modifications to a file in a non-representative format, the MSU Graphics and Media Lab has very precisely implemented the necessary algorithms and documented a few cases where something identifiable has gone wrong.
	
	\cite{ramachandran_phantoms_1999}
	In ``Phantoms in the Brain'', V.S. Ramachandran espouses the notion of looking at where things break to understand how they work. When many patients have lesions in a similar area of the brain, and they all exhibit similar disabilities, it is assumed that that region of the brain is responsible for the missing behavior. More interesting than behavior localization is exploring the mechanisms for normal operation: the idea of a malleable body-image and it's ``phantom limb'' glitch, the notion of ``filling-in'' happening regularly in our blindspots at the optic nerve (or sometimes more extremely in the case of cataracts and blindsight), etc. The brain can experience glitches that reveal its inner workings the same way any other media might.
	
\chapter{Only Everything Lasts Forever}
\section{The History and Psychoacoustics of MP3}
Lossy audio compression taking digital information that describes an audio signal and representing it with less information without producing a significantly different sound. This is called ``perceptual encoding''.\cite{Ruckert05}

Karlheinz Brandenburg's PhD dissertation formed the foundation for the MP3 format. His advisor was interested in transmitting music over telephone lines(ISDN), but the patent office said it was impossible. MP3 was 20 years in development. When asked early on ``What will happen to this?'' he replied ``It could end up in the libraries, like so many other theses, or it could be an international standard.'' He adds: ``I didn't dream of hundreds of millions of people.'' Fraunhofer knew they wanted to make MP3 ``the'' internet format, and took that opportunity, but didn't realize the full extent of what that meant. Now he's looking into WFS and better DRM solutions.\cite{brandenburg_interviews_2004}
	
Brandenburg admits that there are ``compromises'' in MP3, due to needing to inherit the format of MP2. ``Dry percussive material'' doesn't translate well (pre-echo effect most obvious when listening to castanets, which is not a problem with AAC). Low bitrates have ``bandpass problems at high frequencies'' which he vocally imitates as ``shw-shw-shw''. As a trained listener, he could hear the difference at 192 kbps in a blind test, but can't anymore (``I'm too old''). AAC is what MP3 ``should have been''. MPEG (motion picture experts group) was formed to put video on CD-ROMs, when at the time you had only audio on CD-ROMs (massive compression problem). Layer 2 and Layer 3 were two competing proposals during the MPEG audio format ``shootout'', and were decided to be combined. First PC-based decoders were in 1995. July 14th, 1995 they decided on the file extension ``.mp3'' For a psychoacoustic model he mentions frequency quantization and masking (In \cite{karlheinz_brandenburg_mp3_1999} he describes three ``common types of artifacts'': loss of bandwidth, pre-echoes, and roughness or double-speak.) For masking he gives the example of a train arriving at a station drowning out the conversations around you.

He says how, if you look at the data, you see runs of zeros or ones (that might be compressed by lossless compression routines like RLE), and that ``it's really a miracle with how that is reconstructed to get you the music again'' he says with a smile. He localizes different bitrates to different situations: ``128 might be good enough to listen to it on the airplane or the train''. The way the format is structured, the difference between 128 and 192 is much bigger than 256 and 320: after a certain point it takes more work to represent finer details (in \cite[9]{karlheinz_brandenburg_mp3_1999} he calls it the ``sweet spot''). This might be very loosely analogous to music creation: more people than ever are creating music right now and sharing it, perhaps exponentially, but innovation continues at a steady rate. He smiles when he says that vinyl records have artifacts that ``people come to like''. He talks about Gestalt, and how what we hear is ``clearly influenced from what we expect to hear''. Mentions how recordings can be constructed to fool people who would be able to identify their origin. People buying nice cables because they believe the sound quality will improve. He doesn't dismiss it, he says ``it's real, they think it's better, if they're happy with it then I'm fine as well''. He mentions that he knows people who have many TB of MP3s: ``which means years of uninterrupted listening, which means they have no idea what they have''. (For reference, 1 year at 256 kbps is 7.5 TB) \cite{tom_merritt_real_2010}

While PCM encoding tried to replicate the physicality of audio, as represented on vinyl and tape, MP3 reduces the information to a human-oriented format that makes an attempt at universality. In a way, it is a step towards a universal music.

In \cite[9]{karlheinz_brandenburg_mp3_1999}, Karlheinz mentions that we have all become ``expert listeners'' due to the simple fact that we are well trained at listening to compressed audio.

\cite[10]{karlheinz_brandenburg_mp3_1999} recommends limiting the response of an MP3 or AAC encoder to 16 kHz as there ``are some hints'' that there exist listeners who can identify the difference between complex signals above 16 kHz, but ``the full scientific proof has not yet been given.'' This serves as a piece of the MP3-ideology, acts as an ontology of sound, and limits musical practice.
	
\section{Structurally-Informed MP3 Glitching}

	The protagonist of Borges' ``The Aleph'' makes an interesting comment about the difference between how a work is received and how it is justified:
	
	\begin{quote}
	...Daneri's real work lay not in the poetry but in his invention of reasons why the poetry should be admired. Of course, this second phase of his effort modified the writing in his eyes, though not in the eyes of others.
	\end{quote}
	
\chapter{Conclusion}

% The following produces a numbered bibliography where the numbers
% correspond to the \cite commands in the text.
\specialhead{References}
\begin{singlespace}
\bibliographystyle{plain}
\bibliography{original,spiritofnoise,everythingart,glitchart,oelf}
\end{singlespace}

\end{document}
