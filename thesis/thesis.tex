\documentclass{thesis}

\usepackage{hyperref}
\usepackage{url}

\title{Thesis}
 
\author{Kyle McDonald}

\begin{document}

\maketitle

\chapter{Introduction}


\chapter{The Spirit of Noise}
\section{Epistemology: Correspondence vs. Coherence}
	\cite{Blackburn07}\cite{seti_about_????}\cite{david_correspondence_????}\cite{david_horvitz_flickr:_????}
	\cite{david_horvitz_flickr:_????-1}\cite{young_coherence_????}
\section{Philosophy: Interdependent Arising}
	\cite{Hofstadter07}\cite{Koller01}\cite{erik_thiele_tempest_????}\cite{francesco_vianello_reality_????}
	\cite{w._wayt_gibbs_hackers_2009}
\section{Psychology and Nature: Gestalt, Illusion, and Patterns}
	\cite{Moore07}\cite{Doczi81}\cite{Hofstadter01}\cite{robin_mckie_secret_2004}
	\cite{alan_dunning_paul_woodrow_and_morley_hollenberg_einsteins_2008}\cite{alexander_bogomolny_kanizsa_????}
	\cite{brian_dunning_facemars_2008}\cite{dan_paluska_holy_2005}\cite{de_lapparent_slice_1986}
	\cite{jochem_van_der_spek_no_2001}\cite{jonathan_feinberg_haiku_????}\cite{michael_bach_dalmatian_2002}
	\cite{michael_m._ross_natural_2007}\cite{padovan_proportion_1999}\cite{rips_equidistant_1994}
	\cite{weisstein_prime_????}\cite{boston.com_religious_????}
\section{Music: Acontextual Sound}
	\cite{Bruen04}\cite{Vannort06}\cite{Attali85}\cite{Cage61}\cite{Sangild04}\cite{Cascone00}
	\cite{Hegarty02}\cite{Kahn01}\cite{Russolo04}
	
\chapter{Everything Art}
\section{The Historical ``End of Art''}
	\cite{Wright09}\cite{moma_kazimir_2006}\cite{moma_rodchenko_1998}
\section{Permutation and Enumeration}
	\cite{boolos_computability_2002}\cite{borges_aleph_2004}\cite{borges_library_2000}
	\cite{christian_scheib_statics_????}\cite{jim_campbell_end_1996}\cite{john_f._simon_jr._every_????}
	\cite{john_f._simon_jr._given:32_1997}\cite{kyle_mcdonald_nandhopper_2008}\cite{kyle_mcdonald_pppd_2009}
	\cite{leander_seige_imagen_????}\cite{leonardo_solaas_magic_????}\cite{matthew_mirapaul_in_1997}
	\cite{michael_aschauer_8-bit_????}\cite{nattiez_music_1990}\cite{remko_scha_every_2001}\cite{sintron_gods_2003}
	\cite{tomczak_all_2009}\cite{tomczak_hardware-based_2009}\cite{alexander_christiaan_jacob_allrgb_2008}
	\cite{allan_mccollum_shapes_2006}\cite{jem_finer_longplayer_????}\cite{john_cage_as_????}\cite{paul_slocum_pi_2007}
	\cite{brian_whitman_eigenradio_2005}\cite{keith_f._lynch_converting_????}
\section{Empty Art}
	\cite{larry_j_solomon_sounds_1998}
	
\chapter{Glitch Art}
\section{Lossless Compression and Databending}
	\cite{indianropeburn_databenders_????}\cite{liminalmike_flickr:glitch_????}\cite{cory_arcangel_photoshop_2009}
	\cite{media_art_net_media_2010}
\section{Lossy Compression and Datamoshing}
	\cite{!mediengruppe_bitnik_and_sven_knig_download_????}\cite{evan_meaney_ceibas:_2008}\cite{imbecil_mpeg_2004}
	\cite{john_michael_boling_rhizome_????}\cite{nikolai_trunichkin_and_dr._dmitriy_vatolin_crazy_????}
	\cite{ramachandran_phantoms_1999}
	
\chapter{Only Everything Lasts Forever}
\section{The History and Psychoacoustics of MP3}
Lossy audio compression taking digital information that describes an audio signal and representing it with less information without producing a significantly different sound. This is called ``perceptual encoding''.

	\cite{Ruckert05}
Karlheinz Brandenburg's PhD dissertation formed the foundation for the MP3 format. His advisor was interested in transmitting music over telephone lines(ISDN), but the patent office said it was impossible. MP3 was 20 years in development. When asked early on ``What will happen to this?'' he replied ``It could end up in the libraries, like so many other theses, or it could be an international standard.'' He adds: ``I didn't dream of hundreds of millions of people.'' Fraunhofer knew they wanted to make MP3 ``the'' internet format, and took that opportunity, but didn't realize the full extent of what that meant. Now he's looking into WFS and better DRM solutions.\cite{brandenburg_interviews_2004}
	
Brandenburg admits that there are ``compromises'' in MP3, due to needing to inherit the format of MP2. ``Dry percussive material'' doesn't translate well (pre-echo effect most obvious when listening to castanets, which is not a problem with AAC). Low bitrates have ``bandpass problems at high frequencies'' which he vocally imitates as ``shw-shw-shw''. As a trained listener, he could hear the difference at 192 kbps in a blind test, but can't anymore (``I'm too old''). AAC is what MP3 ``should have been''. MPEG (motion picture experts group) was formed to put video on CD-ROMs, when at the time you had only audio on CD-ROMs (massive compression problem). Layer 2 and Layer 3 were two competing proposals during the MPEG audio format ``shootout'', and were decided to be combined. First PC-based decoders were in 1995. July 14th, 1995 they decided on the file extension ``.mp3'' For a psychoacoustic model he mentions frequency quantization and masking (In \cite{karlheinz_brandenburg_mp3_????} he describes three ``common types of artifacts'': loss of bandwidth, preechoes, and roughness or double-speak.) For masking he gives the example of a train arriving at a station drowning out the conversations around you.

He says how, if you look at the data, you see runs of zeros or ones (that might be compressed by lossless compression routines like RLE), and that ``it's really a miracle with how that is reconstructed to get you the music again'' he says with a smile. He localizes different bitrates to different situations: ``128 might be good enough to listen to it on the airplane or the train''. The way the format is structured, the difference between 128 and 192 is much bigger than 256 and 320: after a certain point it takes more work to represent finer details (in \cite[9]{karlheinz_brandenburg_mp3_????} he calls it the ``sweet spot''). This might be very loosely analogous to music creation: more people than ever are creating music right now and sharing it, perhaps exponentially, but innovation continues at a steady rate. He smiles when he says that vinyl records have artifacts that ``people come to like''. He talks about Gestalt, and how what we hear is ``clearly influenced from what we expect to hear''. Mentions how recordings can be constructed to fool people who would be able to identify their origin. People buying nice cables because they believe the sound quality will improve. He doesn't dismiss it, he says ``it's real, they think it's better, if they're happy with it then I'm fine as well''. He mentions that he knows people who have many TB of MP3s: ``which means years of uninterrupted listening, which means they have no idea what they have''. (For reference, 1 year at 256 kbps is 7.5 TB) \cite{tom_merritt_real_2010}

While PCM encoding tried to replicate the physicality of audio, as represented on vinyl and tape, MP3 reduces the information to a human-oriented format that makes an attempt at universality. In a way, it is a step towards a universal music.

In \cite[9]{karlheinz_brandenburg_mp3_????}, Karlheinz mentions that we have all become ``expert listeners'' due to the simple fact that we are well trained at listening to compressed audio.

\cite[10]{karlheinz_brandenburg_mp3_????} recommends limiting the response of an MP3 or AAC encoder to 16 kHz as there ``are some hints'' that there exist listeners who can identify the difference between complex signals above 16 kHz, but ``the full scientific proof has not yet been given.'' This serves as a piece of the MP3-ideology, acts as an ontology of sound, and limits musical practice.
	
\section{Structurally-Informed MP3 Glitching}

\chapter{Conclusion}

\bibliographystyle{plain}
\bibliography{original,spiritofnoise,everythingart,glitchart,oelf}

\end{document}